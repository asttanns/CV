
\documentclass[11pt]{article}
\usepackage[margin=1in]{geometry}
\usepackage[T1]{fontenc}
\usepackage{lmodern}
\usepackage[hidelinks]{hyperref}
\usepackage{enumitem}
\usepackage{titlesec}
\usepackage{setspace}

% -------------------- Spacing & Lists --------------------
\setlength{\parindent}{0pt}
\setlength{\parskip}{0.35\baselineskip} % consistent paragraph spacing
\setlist[itemize]{leftmargin=*, itemsep=0.15\baselineskip, topsep=0.15\baselineskip}
\setlist[enumerate]{leftmargin=*, itemsep=0.15\baselineskip, topsep=0.15\baselineskip}

% -------------------- Section formatting --------------------
\titleformat{\section}{\large\bfseries}{}{0pt}{}
\titlespacing*{\section}{0pt}{0.25\baselineskip}{0.25\baselineskip}

% -------------------- Hanging indent macro for publications --------------------
\newcommand{\pub}[1]{\par\hangindent=1.6em\hangafter=1 #1\par}

% -------------------- Name block --------------------
\begin{document}

{\LARGE \textbf{An Nisa Astuti}}\\
\href{mailto:annisaastuti@uchicago.edu}{annisaastuti@uchicago.edu}

\section*{Education}
\textbf{M.A., Computational Social Science}, University of Chicago \hfill Expected 2027

\textbf{B.Pol.Sci.,} University of Indonesia \hfill 2018\\
Thesis: \emph{The Gender Dimension of Social Movements: Peasant Women’s Position in Resistance Against a Cement Factory in the North Kendeng Mountains, Rembang, Central Java}

\section*{Publications}

\textbf{Book chapter}
\pub{Fajar, Muhammad, \textbf{An Nisa Astuti}, \& Carolus Bregas Pranoto. 2024. ``Progressive Yet Powerless: The State of Indonesia’s Progressive Youth Organizations in the Post-Authoritarian Era.'' In \emph{Understanding the Role of Indonesian Millennials in Shaping the Nation’s Future: Lectures, Workshops, and Proceedings of International Conferences}, eds. Ju-Lan Thung \& Maria Monica Wihardja, 70--98. Singapore: ISEAS--Yusof Ishak Institute / Cambridge University Press.}

\textbf{Research reports / monographs}
\pub{Fajar, M., \textbf{A. N. T. Astuti}, \& C. B. Pranoto. 2021. \emph{The Variety of Indonesian Progressive Youth Organizations}. Jakarta: Collective for Action and Mobilization Studies.}
\pub{Fajar, M., \textbf{A. N. T. Astuti}, \& C. B. Pranoto. 2019. \emph{The Kamisan Movement: Case Studies in Eight Cities}. Jakarta: Victim Solidarity Network for Justice (JSKK) \& Collective for Action and Mobilization Studies.}

\textbf{Policy report}
\pub{Haristya, S., S. Laksmi, \textbf{A. N. T. Astuti}, \& I. F. Dewi. 2020. \emph{Preliminary Study: A Comparison of Indonesia’s Personal Data Protection Bill with the Council of Europe’s Convention 108+ (CoE 108+) and the General Data Protection Regulation (GDPR)}. Jakarta: Tifa Foundation.}

\textbf{Op-ed / public writing}
\pub{Fajar, M., \textbf{A. N. T. Astuti}, \& C. B. Pranoto. 2020. ``Three Lessons from the Kamisan Protest in Building Social Movements.'' \emph{The Conversation} (6 Aug.). \url{https://theconversation.com/tiga-pelajaran-dari-aksi-kamisan-dalam-membangun-gerakan-sosial-143445}}

\section*{Works in Progress}
\textbf{Adapting to Adversity: The Variety of Indonesian Human Rights Organizations’ Capacity to Respond to Repression.}

\section*{Research \& Professional Experience}

\textbf{Program Officer, Amnesty International Indonesia} \hfill 2022--2025
\begin{itemize}
    \item Led scoping research with LabSosio (University of Indonesia) on the consequences of repression on social movement organizations and their capacity to respond. (online survey $n=101$; in-depth interviews $n=15$)
    \item Manage Collective Action Lab initiative; current project studies activists’ perceptions of movement success at different career stages.
\end{itemize}

\textbf{Researcher, KOALISI -- Collective for Action and Mobilization Studies} \hfill 2019--present
\begin{itemize}
    \item Mixed-methods research on Indonesian progressive youth organizations (online survey $n=207$; in-depth interviews $n=81$).
    \item Conducted Qualitative Comparative Analysis (QCA) on the success and durability of the Kamisan Protest across eight cities.
\end{itemize}

\textbf{Program Assistant, Tifa Foundation} \hfill 2020--2021
\begin{itemize}
    \item Co-produced comparative research on Indonesia’s Personal Data Protection Bill vs. CoE 108+ and GDPR.
\end{itemize}

\section*{Conference Presentations}
``Scaling Success Among Indonesian Civil Society,'' co-presented at \textbf{Indonesia Council Open Conference}, SSEAC, University of Sydney, Sept.\ 26--27, 2023.

``The Variety of Indonesian Progressive Youth Organizations,'' co-presented at \textbf{Conference on Millennial Disruptions}, ISEAS--Yusof Ishak Institute, Aug.\ 15--16, 2022.

\section*{Training}
Python for Data Science, Pacmann Academy, Mar.\ 2024.

KELAS Reading Seminar: Democratic Decline \& Social Movement Strategy, Atma Jaya Institute for Advanced Research, Oct.\ 2024.

KELAS Reading Seminar: Dynamics of Organizations, Atma Jaya Institute for Advanced Research, Oct.\ 2023.

\section*{References}
Available upon request.

\end{document}
